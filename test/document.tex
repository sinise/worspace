\documentclass[12pt]{article}
\usepackage{amsmath} % flere matematikkommandoer
\usepackage[utf8]{inputenc} % æøå
\usepackage[T1]{fontenc} % mere æøå
\usepackage[danish]{babel} % orddeling
\usepackage{enumerate}

\title{Projekt A}
\author{Sebastian O. Jensen
		GJX 653
		hold 3}

\begin{document}
\newcommand{\Eone}{
\left(\begin{array}{ccc}
1&0&0
\\
-4&1&0
\\
0&0&1
\end{array}\right)
  }
  
\newcommand{\Etwo}{  
\left(\begin{array}{ccc}
1&0&0
\\
0&0&1
\\
0&1&0
\end{array}\right)
}

\newcommand{\Ethree}{
\left(\begin{array}{ccc}
1&0&0
\\
0&1&0
\\
0&0&1/5
\end{array}\right)
}

\newcommand{\Efour}{
\left(\begin{array}{ccc}
1&0&1
\\
0&1&0
\\
0&0&1
\end{array}\right)
}

\newcommand{\fourXthree}{
\left(\begin{array}{ccc}
1&0&1/5
\\
0&1&0
\\
0&0&1/5
\end{array}\right)
  }
  
\newcommand{\fourXthreeXtwo}{
\left(\begin{array}{ccc}
1&1/5&0
\\
0&0&1
\\
0&1/5&0
\end{array}\right)
  }  
  
\newcommand{\fourXthreeXtwoXone}{
\left(\begin{array}{ccc}
1/5&1/5&0
\\
0&0&1
\\
-4/5&1/5&0
\end{array}\right)
  }  

\newcommand{\fiveXfourXthreeXtwoXone}{
\left(\begin{array}{ccc}
1&1&0
\\
0&0&1
\\
-4&1&0
\end{array}\right)
  }  
  
\maketitle
\section{Opgave 1}
\begin{enumerate}[(a)]
\item Jeg opskriver totalmatricen for ligningsystemet
$$
\mathbf A=
\left(\begin{array}{cccc}
2&3&-1&2\\1&1&1&1\\4&-1&a&4
\end{array}\right)
$$

	Jeg fortager nu de nævnte rækkeoperationer.
	 
	$	r_{1} \leftrightarrow r_{2}$ 
$$
\mathbf A=
\left(\begin{array}{cccc}
1&1&1&1
\\
2&3&-1&2
\\
4&-1&a&4
\end{array}\right)
$$
	
		$r_{2} - 2 \cdot r1$

$$
\mathbf A=
\left(\begin{array}{cccc}
1&1&1&1
\\
0&1&-3&0
\\
4&-1&a&4
\end{array}\right)
$$
		
		$r_{3} - 4 \cdot r1$

$$
\mathbf A=
\left(\begin{array}{cccc}
1&1&1&1
\\
0&1&-3&0
\\
0&-5&a-4&0
\end{array}\right)
$$

		$r_{3} + 5 \cdot r2$

$$
\mathbf A=
\left(\begin{array}{cccc}
1&1&1&1
\\
0&1&-3&0
\\
0&0&a-19&0
\end{array}\right)
$$
Den omformede matrice er på rækkeechelonform da den første pivot i vær række
ligger til højre for pivoten i rækken oven for, samtidig skal evt. nul rækker ligge under alle ikke nul
rækker. Den omformede matrice er ikke på reduceret echelonform da ikke alle
pivot’er 1 samt at der over pivot'erne ikke står 0.



\item Jeg lader nu a = 19 og opskriver matricen med den kendte værdi for
$$
\mathbf A=
\left(\begin{array}{cccc}
1&1&1&1
\\
0&1&-3&0
\\
0&0&19-19&0
\end{array}\right)
$$

$$
\mathbf A=
\left(\begin{array}{cccc}
1&1&1&1
\\
0&1&-3&0
\\
0&0&0&0
\end{array}\right)
$$

Jeg bringer nu matricen på reduceret rælleechelonform vha. backward reduction
	
	$	r_{1} - r_{2}$
$$
\mathbf A=
\left(\begin{array}{cccc}
1&0&4&1
\\
0&1&-3&0
\\
0&0&0&0
\end{array}\right)
$$

Løsningen til ligningssytemet er hermed givet
 
	$	(x_{1}, x_{2}, x_{3}) = (1-4t, +3t, t)$

\item
Jeg lader nu a = 20 og bestemer den inverse matrix til koefficientmatricen.


$$
\mathbf A \cdot I_{3} =
\left(\begin{array}{ccc|ccc}
1&1&1&1&0&0
\\
0&1&-3&0&1&0
\\
0&0&1&0&0&1
\end{array}\right)
$$

$	r_{1} - r_{2}$

$$
\mathbf A \cdot I_{3} =
\left(\begin{array}{ccc|ccc}
1&0&4&1&-1&0
\\
0&1&-3&0&1&0
\\
0&0&1&0&0&1
\end{array}\right)
$$

$	r_{1} - 4r_{3}$

$$
\mathbf A \cdot I_{3} =
\left(\begin{array}{ccc|ccc}
1&0&0&1&-1&-1
\\
0&1&-3&0&1&0
\\
0&0&1&0&0&1
\end{array}\right)
$$

$	r_{2} + 3r_{3}$

$$
\mathbf A \cdot I_{3} =
\left(\begin{array}{ccc|ccc}
1&0&0&1&-1&-1
\\
0&1&0&0&1&3
\\
0&0&1&0&0&1
\end{array}\right)
$$

Den inverse matrix X er her med bestemt til følgende


$$
\mathbf X =
\left(\begin{array}{ccc}
1&-1&-1
\\
0&1&3
\\
0&0&1
\end{array}\right)
$$
\end{enumerate}
\newpage
\section{Opgave 2}
\begin{enumerate}[(a)]
  \item Jeg bestemer de elementære matricer ud fra de givne ero.

$$\mathbf E_{1} =  \Eone $$

$$\mathbf E_{2} = \Etwo $$

$$\mathbf E_{3} = \Ethree $$

$$\mathbf E_{4} = \Efour $$

\item
Jeg betsemer nu F ved at gange de elementære matricer sammen

$$ F = \Efour \cdot \Ethree \cdot \Etwo \cdot \Eone $$

$$ = \fourXthree \cdot \Etwo \cdot \Eone$$

$$ = \fourXthreeXtwo \cdot \Eone$$

$$ = \fourXthreeXtwoXone$$

$$ = 5 \cdot \fourXthreeXtwoXone$$

$$ = \fiveXfourXthreeXtwoXone$$

F er hermed bestemt.

Jeg finder nu de inverse matricer til de elementære matricer ved at benytte de
inverse ERO og benyte dem på $I_{2}$


$r_{2} + 4\cdot r_{1} $ $$E_{1}^{-1} = \left(\begin{array}{ccc}
1&0&0
\\
4&1&0
\\
0&0&1
\end{array}\right) $$
  
$ r_{2} \leftrightarrow r_{3} $ $$ E_{2}^{-1} = \left(\begin{array}{ccc}
1&0&0
\\
0&0&1
\\
0&1&0
\end{array}\right) $$

$ r_{3} \cdot 5 $ $$ E_{3}^{-1} = \left(\begin{array}{ccc}
1&0&0
\\
0&1&0
\\
0&0&5
\end{array}\right) $$

$ r_{1} - r_{3}$ $$ E_{4}^{-1} = \left(\begin{array}{ccc}
1&0&-1
\\
0&1&0
\\
0&0&1
\end{array}\right) $$

Jeg bestemer nu G ved at gange matricerne sammen fra venstre og får følgende


$$G=E_{1}^{-1}E_{2}^{-1}E_{3}^{-1}E_{4}^{-1}$$

$$
\left(\begin{array}{ccc}
1&0&0
\\
4&1&0
\\
0&0&1
\end{array}\right)
\left(\begin{array}{ccc}
1&0&0
\\
0&0&1
\\
0&1&0
\end{array}\right) 
\left(\begin{array}{ccc}
1&0&0
\\
0&1&0
\\
0&0&5
\end{array}\right) 
\left(\begin{array}{ccc}
1&0&-1
\\
0&1&0
\\
0&0&1
\end{array}\right) = 
\left(\begin{array}{ccc}
1&0&-1
\\
4&0&1
\\
0&1&0
\end{array}\right)$$

\item
Fra bogen har jeg at den inverse matrix $A^{-1}$ kan udtrykes ved
produktet af de elementære matricer som svare til de ero som omformer A til
enhedsmatricen. Det er lige netop hvad F er udtryk for i opgave b. Også fra
bogen har jeg at den ``ikke'' inverse matrice af F kan udtrykes som følgende
$$A=E_{1}^{-1}E_{2}^{-1}E_{3}^{-1}E_{4}^{-1}$$ Hvilket svare til matricen G.
Derfor er A givet ved følgende matrice
$$\left(\begin{array}{ccc}
1&0&-1
\\
4&0&1
\\
0&1&0
\end{array}\right)$$


\end{enumerate}
\section{Opgave 3}
\begin{enumerate}[{a}]
\item
Jeg bestemer nabomatricen
$$N = \left(\begin{array}{ccccc}
0&1&1&0&1
\\
1&0&0&0&1
\\
0&1&0&1&0
\\
0&0&1&0&0
\\
1&0&0&0&0
\end{array}\right)$$

Da jeg har oplyst $N^{6}$ benytter jeg denne til at at aflæse antalet af veje
fra knude 2 til knude 2 af længden 6. Denne værdi er angivet på plads
${N^{6}}_{22}$ og har værdien 12. Der er derfor 12 veje fra kude 2 til kunde 2.

\item
Lad $N_{j}$ være antalet af links der udgår fra knude j. Så har jeg følgende
$$N_{1} = 3, N_{2} = 2, N_{3} = 2, N_{4} = 1, N_{5} = 1$$

Jeg kan nu besteme ligningerne for $x_1, x_2, x_3, x_4, x_5, $

$$ x_1 = x_5 + 1/2 x_2$$
$$ x_2 = 1/2x_3 + 1/3x_1$$
$$ x_3 = 1/3x_1 + x_4$$ 
$$ x_4 = 1/2x_3$$
$$ x_5 = 1/3x_1 + 1/2x_2$$

Jeg kan nu besteme linkmatricen A Til følgende

$$A = \left(\begin{array}{ccccc}
0&1/2&0&0&1
\\
1/3&0&1/2&0&0
\\
1/3&0&0&1&0
\\
0&0&1/2&0&0
\\
1/3&1/2&0&0&0
\end{array}\right)$$

\item
Linkmatricen kan skrives på formen $Ax=x$ hvor

$$x = \left(\begin{array}{c}
x_1\\x_2\\x_3\\x_4\\x_5
\end{array}\right)$$

Dette kan omformes til følgende total matrice
$$A = \left(\begin{array}{ccccc|c}
-1&1/2&0&0&1
&0\\
1/3&-1&1/2&0&0
&0\\
1/3&0&-1&1&0
&0\\
0&0&1/2&-1&0
&0\\
1/3&1/2&0&0&-1&0
\end{array}\right)$$

Jeg benytter mig af Gaussian Elimination og får følgende matrice

$$A = \left(\begin{array}{ccccc|c}
1&0&0&0&-3/2&0
\\
0&1&0&0&-1&0
\\
0&0&1&0&-1&0
\\
0&0&0&1&-1/2&0\\
0&0&0&0&0&0
\end{array}\right)$$

Jeg kan nu bestemme vektoren x til følgende
$$x = \left(\begin{array}{c}
3/2\\1\\1\\1/2\\1
\end{array}\right)$$

Sider kan nu rangeres. $x_1$ har højest rang $x_2, x_3 $ og $ x_5$ har samme
rang og kommer lige efter $x_1$. $x_5$ har laveste rang.

\end{enumerate}
\end{document}
