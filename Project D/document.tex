%%This is a very basic article template.
%%There is just one section and two subsections.
\documentclass[12pt]{article}
\usepackage{amsmath} % flere matematikkommandoer
\usepackage[utf8]{inputenc} % æøå
\usepackage[T1]{fontenc} % mere æøå
\usepackage[danish]{babel} % orddeling
\usepackage{enumerate}
\usepackage{tabto}
\title{Projekt C}
\author{Sebastian O. Jensen, GJX 653}

\begin{document}

\newcommand{\aar}{
1  }
\newcommand{\ab}{
3  }
\newcommand{\ac}{
1  }
\newcommand{\ba}{
4  }
\newcommand{\bb}{
2  }
\newcommand{\bc}{
-1  }
\newcommand{\ca}{
-2  }
\newcommand{\cb}{
4 }
\newcommand{\cc}{
2  }

\maketitle
\newpage
\section{Opgave}
\begin{enumerate}[(a)]
\item 
Det karakteristiske polynomie $p(\lambda)$ for A er givet ved
$$
p(\lambda) = det(\lambda I-A) = \begin{array}{|ccc|}
\lambda-3&-1&0\\
-6&\lambda-2&0\\
0&0&\lambda+1
\end{array}
$$
$$
p(\lambda) = (\lambda-3) (\lambda-2) (\lambda+1) +
(1\cdot 0 \cdot 0)
+ (0 \cdot (-6) \cdot 0)
-
$$$$(0 \cdot (\lambda -2)  \cdot 0)
-(0 \cdot 0 \cdot (\lambda -3))
-((\lambda +1)(-6)(-1))= 
$$
$$
(\lambda-3) (\lambda-2) (\lambda+1)
-((\lambda +1) \cdot (-6)\cdot (-1))=
$$$$
(\lambda-3) (\lambda-2) (\lambda+1)
-(6\lambda +6)=
$$$$
(\lambda^2-5\lambda +6)(\lambda+1)-(6\lambda +6)=
$$
$$
\lambda^3 + \lambda^2 -5\lambda^2 -5\lambda +6\lambda + 6
-6\lambda -6= $$
$$
\lambda^3 -4\lambda^2 -5\lambda
= (\lambda)(\lambda+1)(\lambda-5) $$

Hermed har vi fundet det karakteristiske poynomie
\item
Fra faktoriseringen i opgave a kan jeg aflæsse røderne til det karakteristiske
polynomie til at være 0 -1 og 5, hvilket derfor er eigenværdierne for A

\item 
For vær eigenværdi findes den reducerede matrice af $A-\lambda I_3$ hvoraf
eigenvektorene aflæsses $$
A+1I_3 = \left(\begin{array}{ccc}
4&1&0\\
6&3&0\\
0&0&0
\end{array}\right)\sim
\left(\begin{array}{ccc}
1&0&0\\
0&1&0\\
0&0&0
\end{array}\right) \Rightarrow E_-1 = t
\left(\begin{array}{ccc}
0\\
0\\
1
\end{array}\right)
$$

$$
A-0 I_3 = \left(\begin{array}{ccc}
3&1&0\\
6&2&0\\
0&0&1
\end{array}\right)\sim
\left(\begin{array}{ccc}
1&1/3&0\\
0&0&1\\
0&0&0
\end{array}\right) \Rightarrow E_-1 = t
\left(\begin{array}{ccc}
-1\\
3\\
0
\end{array}\right)
$$

$$
A-0 I_3 = \left(\begin{array}{ccc}
-2&1&0\\
6&-3&0\\
0&0&-6
\end{array}\right)\sim
\left(\begin{array}{ccc}
1&-1/2&0\\
0&0&1\\
0&0&0
\end{array}\right) \Rightarrow E_-1 = t
\left(\begin{array}{ccc}
1\\
2\\
0
\end{array}\right)
$$

$$
A-\lambda I_3 = \left(\begin{array}{ccc}
-(\lambda-3)&1&0\\
6&-(\lambda-2)&0\\
0&0&-(\lambda+1)
\end{array}\right) \Rightarrow E_-1 = t
\left(\begin{array}{ccc}
0\\
0\\
1
\end{array}\right)
$$
Hermed er de 3 eigenvektore fundet.
\end{enumerate}
\section{Opgave}
\begin{enumerate}[(a)]
\item
Determinanten af koefficientmatricen findes ved følgende udtryk
$$
a{11}a{22}a{33+}a{12}a{23}a{31+ }a{13}a{21}a{32-
}a{11}a{23}a{32-}a{12}a{22}a{33- }a{13}a{22}a{31}=
$$
$$
 \aar \cdot \bb \cdot \cc+
 \ab \cdot (\bc) \cdot (\ca)+
 \ac \cdot \ba \cdot \cb-
 \aar \cdot (\bc) \cdot \cb-
 \ab \cdot \ba \cdot \cc-
 \ac \cdot \bb \cdot (\ca)=10
$$
\item
Lad koeficientmatricen være givet ved A
$$
det(A_{31}) = 3\cdot (-1)-2 \cdot 1= -5
$$
$$
C_{31} = (-1)^{3+1} \cdot -5= -5
$$
Da $adjA = C^T$ hvor C er kofactor matricen til A vil
$$adjA_{13}= C_{31}=-5$$
Af adjoint formlen følger
$$
A^{-1}=\frac{adjA}{det(A)}
$$
Elementet i den inverse matrix til koeficientmatricen i række 1 søjle 3 er
derfor givet ved følgende
$$
A^{-1}_{13} = \frac{1}{10}\cdot -5=-1/2
$$

\item 
vha. Cramers formel løses ligningssystemet. Først findes determinanten til
$A_1$, $A_2$ og$A_3$

$$
detA_1=\begin{array}{|ccc|}
0&3&1\\
0&2&-1\\
1&4&2
\end{array} = -5
$$

$$
detA_2=\begin{array}{|ccc|}
1&0&1\\
4&0&-1\\
-2&1&2
\end{array}=5
$$

$$
detA_2=\begin{array}{|ccc|}
1&3&0\\
4&2&0\\
-2&4&1
\end{array}=-10
$$
$$
x_1 = \frac{|A_1|}{|A|}=\frac{-5}{10}=-1/2
$$
$$
x_2 = \frac{|A_1|}{|A|}=\frac{5}{10}=1/2
$$
$$
x_3 = \frac{|A_1|}{|A|}=\frac{-10}{10}=-1
$$
ligningsystemet er hermed løst.
\item
Der gælder følgende
$$det(EA)=det(E)det(A=)$$
$$det(A^T)=det(A)$$
$$det(A_)=-det(A)$$

Hvor $A_1$ er A hvor to rækker er ombyttet
Determinanten af matrixproduktet er derfor givet ved følgende

$$10\cdot (-10)\cdot 10 = 1000$$ 

\end{enumerate}
\section{Opgave}
\begin{enumerate}[(a)]
\item

\begin{enumerate}[(i)]
  \item 
$$
HB1 =\left(\begin{array}{ccc}
0&1&0\\
2&0&1\\
2&-1&1
\end{array}\right)\left(\begin{array}{ccc}
7&12&3\\
4&4&8\\
3&3&6
\end{array}\right)=
\left(\begin{array}{ccc}
4&4&8\\
17&27&12\\
13&23&4
\end{array}\right)
$$

$$
HB2 =\left(\begin{array}{ccc}
0&1&0\\
2&0&1\\
2&-1&1
\end{array}\right)
\left(\begin{array}{ccc}
7&12&2\\
4&4&8\\
3&3&8
\end{array}\right)=
\left(\begin{array}{ccc}
4&4&8\\
17&27&12\\
13&23&4
\end{array}\right)
$$
\end{enumerate}
\end{document}
