\documentclass[a4paper,12pt,danish]{report}
\usepackage[utf8]{inputenc}
\usepackage[danish]{babel}
\usepackage{amsmath}
\usepackage{amssymb}
\usepackage{listings}
\usepackage{hyperref}
\usepackage{booktabs}
\usepackage{graphicx}
\usepackage{makeidx}
\usepackage{titlesec}
\usepackage{fancyhdr}
\usepackage{wrapfig}
\usepackage{fancyvrb}
\usepackage{pbox}
\usepackage{hyperref}
\usepackage{mathtools}
\usepackage{amsmath}
\usepackage{multicol}
\pagestyle{fancy}
%Setting link borders to none
\hypersetup{pdfborder = {0 0 0}}

\fancyhead[C]{}
\fancyhead[L]{}
\fancyhead[R]{\footnotesize{
Sebastian O. Jensen \textsc{140679}}}

\begin{document}
\begin{titlepage}

\newcommand{\HRule}{\rule{\linewidth}{0.4mm}}
\center
\small{14.06-79 Sebastian Ostenfeldt Jensen \textsc{GJX653}
} \\[2cm]

\textsc{\LARGE Datalogisk Institut}\\[0.5cm]
\textsc{\large Københavns Universitet}\\[1.5cm]
\textsc{\large Operating Systems and Concurrent Programming}\\
\HRule \\[0.7cm]
{\huge \bfseries 2015}\\[0.4cm]
\HRule \\[1.5cm]
\textsc{\Large \textsc{\today}}\\[0.5cm]

\includegraphics[scale=0.5]{ku_logo.png}\\[1cm]

\end{titlepage}
\tableofcontents
\newpage
\renewcommand{\thesection}{\arabic{section}}
\renewcommand{\thempfootnote}{\arabic{mpfootnote}}
\renewcommand\thesubsection{}
\newcommand{\minus}[1]{{#1}^{-}}
\section{P1 benchmarking: Thread-safe queues}
The two given implementations use a linked list as a
data structure for the queue. The main difference is that fg\_queue uses fine-grained locking and
cg\_queue uses coarse grained locking, when adding and remove elements from the
queue. When fine-grained locking is used the head and tail are locked
independently. Therefore can the head and tail be accesssed at the same time.
E.g. a thread can remove an element at the same time another thread adds an
element. When coarse grained locking is used a global lock on the hole queue is used.
This result in only one thread can accesss the queue at a time.
//
To benchmark the speed of the different implementations it is needed to let
multiple threads insert and remove elements at the same time under different
circumstances. A benchmark has been implemented in the folder benchmark in
the handed in source tree. To start the benchmark one can under Linux issue
the command ``make benchmark".
First the benchmark will measure the time it takes to insert and remove 1.000.000
elements width fine and coarse-grained locking. This is done with one thread
adding elements while another thread remove elements.
If the queue becomes empty before the thread that removes elements has
removed 1000.000 elements the thread will  make a loop and try again
until all 1000.0000 elements has been removed. This could theoretically happen
if the adding thread is not adding elements as fast as the removing thread
consumes them. The same test is then made using 4 threads, where 2 threads are
adding and 2 threads removing elements. Then we move to using 8 threads (4
adding and 4 removing). All this is then repeated with 2-, 3-, 4-, 5-, 10-
millions items.
The output to the console is as follow.
\begin{lstlisting}

                         |        time         |
Threads        items      fine          coarse
 2          | 1000000 |    0.41s        0.28s
 4          | 1000000 |    0.46s        0.43s
 8          | 1000000 |    0.71s        1.00s

 2          | 2000000 |    0.70s        0.34s
 4          | 2000000 |    0.84s        0.80s
 8          | 2000000 |    0.79s        1.98s

 2          | 3000000 |    0.96s        1.49s
 4          | 3000000 |    1.17s        1.28s
 8          | 3000000 |    1.18s        2.95s

 2          | 4000000 |    1.27s        2.19s
 4          | 4000000 |    1.41s        1.49s
 8          | 4000000 |    1.54s        4.04s

 2          | 5000000 |    0.89s        2.71s
 4          | 5000000 |    1.67s        2.28s
 8          | 5000000 |    1.73s        4.92s

 2          | 10000000 |    2.89s        4.82s
 4          | 10000000 |    3.49s        4.38s
 8          | 10000000 |    3.51s        9.97s

\end{lstlisting}

Threads is the total number of threads. E.g. 4 threads means 2 threads
adding and 2 threads removing.
Items is the total number of threads to be inserted. So if there are 4 threads
and 1.000.000 items each adding thread adds 500.000 items and each removing
thread removes 500.000 items.
Fine is the time used with fine-grained locking.
Coarse is the time used with coarse grained locking.

As we see fine-grained locking is in all cases faster than coarse-grained. This
is what we can expect as fine grained locking allow an adding and
removing thread to accesss the queue at the same time. What is noticeable is that
using more than 2 threads make the job slower with fine-grained locking. The
test is done on a eight core CPU so there are enough cores to facilitate the
threads. The reason is that no more than 2 cores can accesss the queue at the
same time. So using more threads just adds overhead as nearly the only thing the
threads are doing are accesssing the queues. If the threads were to do other stuff like
heavy calculation in between accesssing the queues, then more than 2 threads
would result in an increase in speed. We especially see a lot of overhead when
using couarse-grained locking and 8 threads. Here there will most of the time be
up to 7 threads waiting to accesss the queue. One other interesting finding is
that using 4 threads on coarse grained locking are in some cases a little faster
than using 2 threads. A conservative guess could be that when using 4
threads it will happen often that when a thread accesss the list the previous
accesss to the list was to the same end. and therefore the address is still in
the CPU's register. E.g. if 1 thread has just accesssed the tail and updated the
tail, then if the next thread accesssing the list also accesss the tail, then the
data might still be in the CPU's register. But if the previous thread that
accesssed the list was accesssing the head, then there is a bigger chance that the
tail's date was paged out to the memory. And with only 2 threads this might
happen more often than with 4 threads.

The test can be reproduced by invoke make benchmark when in the folder benchmark
on a Linux system.

\newpage
\section{P2. Buenos system calls for basic I/O}
The syscall functionn syscall\_rand should return a pseudo random number between
0 and range -1. Buenos already has kernel functionn \_get\_random that calculate
a pseudo random number from the command line argument to buenos randomseed.
The \_get\_rand is defined in lib/rans.S.
The the syscall syscall\_rand is implemented as follow.
\begin{itemize}
  \item in proc/syscall.h the SYSCALL\_RAND is defined as follow
  \begin{lstlisting}
  	/* Random number*/
	#define SYSCALL_RAND 0x310
  \end{lstlisting}
  \item in proc/syscall.c the according syscall functionn is defined to call the
  \_get\_random to let the kernel produce the number for us. This is the
  definition
  \begin{lstlisting}
  	case SYSCALL_RAND:
	V0 = _get_rand((int) A1);
	break;
  	
  \end{lstlisting}
  	
  \item To give accesss to userland processes to call the new syscall we declare
  the functionn int syscall\_rand(int n); in test/lib.h. The functionn is defined
  in test/lib.c and the only thing it does is to make the syscall with the right
  parameters, get the return value and return it to the userland process. This is
  the code changed in test/lib.c
  \begin{lstlisting}
#ifdef PROVIDE_RANDOM_NUMBER
int syscall_rand(int n)
{
  	return (int)_syscall(SYSCALL_RAND, (uint32_t) n, 0, 0);
}
#endif
  
  \end{lstlisting} 
\end{itemize}

To test the new functionn a test program has been implemented which print 6
pseudo random numbers between 0 and 100. The test program be run by invoking
make rand when in the buenos folder and fyams-term is running in the same directory.

see appendix P2 for a list of all code changed.

\newpage
\section{P3. Process communication: Unidirectional pipes}
\subsection{a Implement system call syscall\_pipe}
To allow userland processes
accesss to the syscall\_pipe the functionn is declared and defined in test/lib.h,
test/lib.c, proc/syscall.h and proc/syscall.c. This is done the same way as in P2.
See appendix P3 for the code. The core of the pipe functionnality is implemented
in proc/pipe.c and proc/pipe.h. proc/pipe.h defines the pipe structure and
declare the functionns for working with the pipes. The structure used for the
a pipe is a circular list. It is implemented as an array of chars with a pointer
to the read end and write end. The pipe has space for 100 elements which is
defined in proc/pipe.h The implementation has room for three pipes but can
easily be extended. The three  pipes are named pipe\_a, pipe\_c, pipe\_d and
are initialized together with a corresponding semaphore in the top of proc/pipe.c.
each pipe also has 2 file descriptors which is associated with writing and
reading from the pipe. The file descriptors or file handlers are defined in
proc/syscal.h. Here follow a detailed description of the implementation
when a functionn is mentioned it is assumed it is located in proc/pipe.c if nothing else is stated.
\begin{itemize}
	\item{pipe\_start}

	When the syscall\_pipe functionn is called,
	pipe\_start is is called. This functionn go true the	pipes until it find a pipe
	that is not in use. If all pipes are in use it returns -1. The way it check if
	a pipe is in use is tp try opening its associated semaphore. If the pipe is in
	use a call to usr\_sem\_open will return NULL as the semaphore would then
	already be open. When a free pipe is found, the assoiciated semaphore is
	opened with the state 0 indicating theres a thread in it. The file descriptor
	array which was supplied as a pointer is updated with the value of the pipes read and write file handlers. There by allowing userland processes accesss to the pipes
	file handlers. What also happens is that the pip is initialized by setting the
	pipes read and write end pointers to 0. Then the semaphore for the pipe is
	signalled so it is ready to be used. See appendix P3 for the code
		 
\end{itemize}
\subsection{b Modify existing system calls}
\begin{itemize}
  \item{io\_read and io\_write}

	To let the userland processes use the pipe functionnality changes
	to the sytcall read and write need some modificationn. As these syscall already make use of the
	functionn io\_write and io\_read in the file proc/io.c it is convenient to
	let these functionns handle what happens when read and write is made to a pipe.
	Both for io\_write and io\_read there is as switch case branch which calls
	appropriate functionns according to the right filhandler. Therefore it was easy
	to just insert some more cases in io\_write and io\_read. E.g. was the following
	cases inserted in io\_write
	\begin{verbatim}
case FILEHANDLE_PIPE_A_WRITE:
    res = pipe_write(file, buffer, length);
    break;
case FILEHANDLE_PIPE_B_WRITE:
    res = pipe_write(file, buffer, length);
    break;
case FILEHANDLE_PIPE_C_WRITE:
    res = pipe_write(file, buffer, length);
    break;
	\end{verbatim}
	Similar cases where inserted in io\_read.
  \item{pipe\_read and pipe\_write}

	When pipe\_write is called it first check what file handle it was called with.
	then it ask for the semaphore for the according pipe. When it get accesss it
	checks if the pipe is full. If its not full it write to the pipe release the
	semaphore and return 0. if the pipe was full it spins until the pipe is
	not full any more. Between each time it checks if the pipe is full the
	semaphore is signalled and acquired to allow a reader to get in. Nearly the same
	happens when pipe\_read is called, except that it now read if its not empty and
	spins if its empty until it is not empty any more. Both functionns return -1
	if the pipe is not in use. These is indicated by the associated semaphore is
	not open. When the functionns try to get accesss to a semaphore that is
	closed the functionn returns -1.
	See appendix P3 for full code of pipe\_read
	and pipe\_write.
  \item{syscall\_close}
  
  	Leting syscall\_close work on unidirectional pipes has not been fully
  	implemented do to time pressure. But it is relatively easy to implement. In
  	pipe.c a functionn pipe\_destroy is defined which only thing it does is to
  	close the associated semaphore of the pipe that was given as argument.
  	to let syscall\_close work on unidirectionalpipe the functionn in io\_close
  	in proc/io.c should be expanded with an if statement that checks if the
  	filedescriptor given is one of the pipes filedescriptor. And if it is then
  	call the pipe\_destroy with the given pipe as argument.
  	\item{child process inherent file descriptors to pipe's}

  	When a fork is made the child will get a full copy of the parents address
  	space. Therefore it automatically happens that the child inherent the file
  	descriptors as these are pointers.
\end{itemize}
\subsection{c Implement additional system support}
\begin{itemize}
  \item{syscall\_dup}

  Again to allow userland processes access to the syscall\_dup functionn it is
  declared and defined in test/lib.h, test/lib.c, proc/syscall.h and proc/syscall.c.
  This is again done the same way as in P2. syscall\_dup is implemented in
  proc/dup.h and proc/dup.h. The way to manage what file descriptors is
  duplicated to what file descriptors a mapping is implemented. The mapping is quit simple. it
  consist of an array of int with the size that there are file descriptors in
  the system. In our case 9. The array is named dup\_map. Each entry in the
  array correspond to a fildescripter. E.g. dup\_map[0] is stdin. The array is
  initialized with the functionn dup\_init which set each entry to the
  number as its index. This means that each filedescriptor points to it self.
  the dup\_map\_init function does this and it is called when buenos start up
  from the init/main.c file. When a syscall\_dup is called the value of the
  filedescriptor being duplicated is set in the index of the file descriptor it is
  being duplicated to. See appendix P3 for the code. When as syscall to
  io\_write or io\_read is called these functionns has been modified to use the
  filedescriptor the given file descriptor points to. eg. if stdin which has
  value 0 has been copied to PIPE\_A\_READ which has value 3, then dup\_map[0] =
  3 and now when we read from stdin we actually read from pipe a.

  \item{syscall\_select}

  This has not been implemented yet du time pressure.  But one way to implement
  would be to let the functionn spin and try to acquire a semaphore on the pipes
  one by one and when it gets in to a semaphore it has found a pipe that is
  ready.
\end{itemize}

For all the files dub.c, dub.h, pipe.c and pipe.h these are added to proc/module
so they are being compile.

\subsection{d Test the implemented functionns}
	To test the implementation pipe1 has been run and is working. Only modification
	that is done is that we copy the fildescripter of stdout back to it self, else
	we were not able to print hello world to the screen after receiving it from the
	pipe. if we did not do this, hello world would obviously just be copied back to
	the pipe.
	
	more test has been implemented in pi1.c. These test are self explaining.
	 
	The test can be run by issuing the command make pipe1 and make pipe2  when in
	the root directory of buenos and fyams term is running
\newpage

{BLANK PAGE}

\newpage
\section{T1. Concurency:Parts must come together}
\subsection{a Pseudo-code}
The pseudo code for the thread functionns frame and wheel are provided in the
file T1.c in the folder T. It is recommendedto look at the file as it provide
nice layout and syntax highlighting if using an editor that provide this for c
code. We use 3 variables and a semaphore for each of them:
\begin{itemize}
  \item int bikeInProduction: which can be either 1 or 0.
  \item arrayList frames: which holds which holds the produced frames
  \item FIFOQueye readyBikes: which is a queue where the bikes are placed when
  they are produced.
  \item When the frame thread want to create a frame it first check that
  bikeInProduction is 0. if it is it is allowed to start production of a frame.
  I does that by setting the bikeInProduction to 1 indicating that it has started
  the production. When the frame is produced the new frame is put in to the list
  of frames and it will the try to create a new frame. As mentioned the
  frame thread can only start production of a frame if bikeInProduction is
  set to 0. if it is 1 then the frame thread will keep spinning and
  checking if it change to 0.
  \item The whell thread will only start production of wheel's if there is a
  bike in production, which means that bikeInProduction is set to. If it is one
  1 the wheel thread will set it to 0 and start producing 2 wheels. When the
  wheels are produced it will try to get the frame from the frames list where
  the frame thread is suposed to put it. If the frame thread has not finished
  the frame jet the whell thread will keep spinning until the frame thats in
  production are added to the frames list.
  \item Then it gets the frame and assemble the bike and put it in the readyBikes
  queue. This queue can be seen as the conection with the outside world where
  the bikes can be collected from.
  \item The wheel thread will then try to produce more wheels.
  \item When ever the wheel thread want to start production of wheels it check to
  see if bikeInProduction is 1. if its 0 then the thread keeps checking until it
  becomes 1 before it start the production.
  \item In this way the frame thread will  never start production of a frame
  unless the whell thread is ready or in production of wheels for the last
  created frame. As well will the wheel thread never start production of wheels
  for which a frame is not produced or about to be produced.
\end{itemize}
  
  
Here follow the code of the two threads.

\begin{verbatim}
	int bikeInproduction = 0;
sem mutex1 = 1;

arrayList frames;
sem mutex2 = 1;

FIFOqueue readyBikes
sem mutex3 = 1;

void frame() {
    /*produce frames*/
    while(1){
        P(mutex1);
        if(bikeInProduction){
            bikeInProduction++;
            V(mutex1);
            frame frame = produceFrame());
            P(mutex2);
            frames.add(frame);
            V(mutex2)  
        } else {
            V(mutex1)
        }
    }
}

void wheel() {
    /*produce whells*/
    while(1){
        P(mutex1);
        if(bikeInProduction == 1){
            /*there is a frame in productio so lets make 2 wheels*/
            bikeInProduktion-- /*we make wheels allow more frames to be produced*/
            V(mutex1);
            wheel w1 = produceWheel();
            wheel w2 = produceWheel();
            P(mutex2);
            while(frames.size == 0){ //spin until there is a frame
                V(mutex2);
                sleep 1 sec;
                P(mutex2)
            }
            /*If we are here there must be a a ready frame, lets assemble*/
            frame frame = frames.remove(frame.size); /*remove the frame and return it*/
            V(mutex2); /*we got the frame so release the lock*/
            bike bike = assemble(frame, w1, w2); //made a bike

            /*add the bike to the pile of finished bikes*/
            P(mutex3)
            readyBikes.add(bike);
            V(mutex3)
            else {
                V(mutex2);
        } else {
            /*we are waiting for a frame to get in production*/
            V(mutex1);
        }
    }
}

int main(){
    pthread44create(frameProducer);
    pthread_create(wheelProducer);
}
	
\end{verbatim}  
\subsection{b Correctness}
Whit the current implementation it is trivial that only usable bikes will be
produced as the wheel thread will create 2 wheels for each frame that goes in to
production as well the frame thread will newer start producing more frames if
the wheel thread has not indicated that it will produce frames for the last
created frame. Therefore the 2 threads will wait for each other so we are not
ending up with to many frames or wheels.
It is trivial that the solution is dead lock free as there are no place in the
code where a thread holds a lock while it wait for the other thread. In all the
cases where a thread waits for the other thread it is spinning in a wild loop and
continuously acquiring and leleasing the semaphore on the variable it waiting on
to be changed. Therefore the other thread will eventually get accesss in to the
semaphore.

\newpage
\section{T2.Online algorithms:Binery body system}
\subsection{a Implementation details}
The pseudo code of the two functionns split and coaleace can be found in the file
T2.c in the folder T. It is recommendedto look at that file instead as it
provides a nicer layout and syntax highlighting if using an editor that provides this for c
code. 
The functionn split is implemented as follow.

\begin{itemize}
  \item First it get the log size of the node and if it is 1, then the note can
  not be split any further and the functionn return NULL
  \item Then if the functionn can be split we get to the else branch. Here we
  first calculate the log size of the notes by subtracting 1 from the
  current log size.
  \item We then create a new node named node2. We then let node2 next
  point to the old node's next and node2 previous point to the old node.
  \item Now is only left to let the old node's next point to node2.
  \item We then return node2, and we are done 
\end{itemize}
Here follow the code for the split functionn
\begin{verbatim} 
struct free_list_node* split(struct free_list_node* node){
    unsigned char old_block_log_size = node->block_descr->log_size;
    if(old_block_size == 1) { // the node can not be split any further
        return NULL;
    }
    else { //split the node
        unsigned char new_block_log_size = old_block_log_size/2; 
        free_list_node node2;
        node2->next = node->next; // point to old node's next
        node2->previous = node;   //point to old node
        node->next = node2;       //point to new node 
        return node1->next;
    }
}
\end{verbatim}

For the functionn coalesce we need to determine the address of the body to the
given node. As the body is just before or after the node we can determine
the address of the body by inverting log $size^{th}$  bit of the node's address.

Here is an example where block 1 and 2 has been split into two blocks of log
size 2. That means we need to invert the $2^{th}$ bit of the address of the
block we wants it body from.
The address of block 1 is 000, inverting $2^{th}$ bit gives 010 which is block
3's address. inverting it again gives block 1
\begin{verbatim}
0 0 0 |		    Block 1
0 0 1 |   
0 1 0 *     Block 2
0 1 1 *
1 0 0 |     Block 3
1 0 1 |
1 1 0 |     
1 1 1 |
\end{verbatim}

With these in place we are now ready to implement the functionn coalesce.

\begin{itemize}
  \item First we store the provided node's log size in a var named log\_size
  \item Then we create a number n which a binary number with the only the
  $log_size^{th}$ bit set to 1. to create the number we use c's build in function
  to shift the number 1 log_size times to the left. Then we XOR the address of
  the node with n and we get the body's address.
  \item we then check if it the body is free. If it is free we arrive at the if
  branch.
  \item We then need to determine which node is first in the list. We do this by
  checking if the body's next point to the node. if it does the body is first in
  the list else it was the node that was first. We then store the node and body
  to the variables old\_first and old_last according to our findings.
  \item Then we can coalesce the nodes. First we create a return node. we then
  set the return node's log size to log size + 1 and its next to old\_last's
  next and its previous to old_first's previous.
  \item It is trivial that the two functionns run in O(1) worst case as there
are no loops and it does not call any functionns that has higher
running time than O(1). Here follow the code of the function coalesce.
\end{itemize}
\begin{verbatim}
struct free_list_node* coalesce(struct free_list_node* node){
    log_size = node->block_descr->log_size

    //get address of the body
    //we can get the address of the body by fliping the bit of node's address
    // at position log_size
    unsigned char n = 1 << log_size; //make an n for the XOR operation
    free_list_node node_body = n^node            //XOR n with node which return the body address


    //check if body is free
    if(!(node_body->block_descr->in_use)){ //its not in use we can free it
        free_list_node old_first:
        free_list_node old_last;

        //we need to determine which node is first in the list
        if(node_body->next == node){    //body is before node in list
            old_first = node_body;
            old_last = node;
        }
        else {                           //node is before body in list
            old_first = node;
            old_last = node_body;
        }

        //now we coalesce the nodes
        free_list_node return_node;
        return_node->block_descr->in_use = 0;               //set the new node to not in use
        return_node->block_descr->log_size = log_size + 1;  //set the new log_size
        return_node->next = old_last->next;                 //set the new next to the old last node's next
        return_node->previous = old_first->previous;        //set the new previous to the old first's previous
        return return_node;
    }
    return NULL; // body was not free return NULL
}
\end{verbatim}

\subsection{b Runtime analysis}
\begin{itemize}
  \item Malloc

When malloc is called on the binary body system it iterates true the free list
until a block that is big enough is found. If it is to big it will split the
node. splitting the node is only O(1) and iterating the list is O(n) therefore
a call to malloc is in worst case O(n) because it might need to iterate the hole
list.

\item Free

Free works like this:
Check if the body is free. if it is the coalesce with the body.
this result in a new free block. with this block we also have to check if this's
block's body is free. If it is free then coalesce with the body. This could
continue all true the hole list of memory resulting in a worst case of O(n).
Its ofcoarse not so likely that this will happen. But in worst case it is O(n)
\end{itemize}
\subsection{c Fragmentation analysis, NOTE: for this part examV1.1 was used}

The size of the body systems arena need to be big enough to hold the memerory
when the blocks in use is devided in the most insufficient way. If it is big
enough malloc want break. There exist a lemma that describes this upper bound as
follow. See reference[1]
\begin{itemize}
  \item Let M be the maximum amount of memory space used by a program and n be
  the smallest allocation blocksize. If T is the total amount of memory storage
  in an insufficient scenario then 
  $$T<=\frac{M^{2}}{4n} + \frac{M}2 - \frac{3n} 4$$
  for all allocators
\end{itemize}
As well there exist a theorem for the lower (also see reference [1]) bound
which is as follow.

\begin{itemize}
  \item 
Let M be the maximum amount of memory space
used by the program and n be the smallest allocation block size.
If S is the minimum amount of memory storage sufficient for
all allocators, then
  $$S<=\frac{M^{2}}{4n} + \frac{M}2 - \frac{n} 4$$
\end{itemize}

M is the maximum allocated memory at any time which correspond to the M in the
examtext. n is the minimum allocated block size which correspond to k in the
examtext. S Will then be the lower bound for the arena and T the upper
bound.

\newpage
\section{T3.Protocols: The fundamentals}

\subsection{}


\newpage
\section{References}
\subsection{}

 [1] IEEE TRANSACTIONS ON COMPUTERS,VOL. 59,NO. 4,APRIL 2010
 
 Upper Bounds for Dynamic Memory Allocation
 
 Yusuf Hasan, Wei-Mei Chen, Member, IEEE, J. Morris Chang,
 
 Senior Member, IEEE, and Bashar M. Gharaibeh.
 
http://www.ece.iastate.edu/~morris/papers/10/ieeetc10.pdf

Page 471

\newpage
\section{Appendix P2 code changes}
\subsection{syscall.h}
\begin{lstlisting}
  	/* Random number*/
	#define SYSCALL_RAND 0x310
  \end{lstlisting}
\subsection{syscall.c}
  \begin{lstlisting}
  	case SYSCALL_RAND:
	V0 = _get_rand((int) A1);
	break;
  \end{lstlisting}
  	
\subsection{test/lib.h}
   \begin{lstlisting}
	int syscall_rand(int n);
   \end{lstlisting} 


\subsection{test/lib.c}
\begin{lstlisting}
#ifdef PROVIDE_RANDOM_NUMBER
int syscall_rand(int n)
{
  	return (int)_syscall(SYSCALL_RAND, (uint32_t) n, 0, 0);
}
#endif
    \end{lstlisting} 

\newpage
\section{Appendix P3  code changes}
\subsection{proc/syscall.h}
\begin{verbatim}
/* Random number*/
#define SYSCALL_RAND 0x310

/* Pipe */
#define SYSCALL_PIPE 0x311
#define SYSCALL_PIPE_DESTROY 0x312

/* DUP */
#define SYSCALL_DUP 0x313

/* Console file handles. */
#define FILEHANDLE_STDIN    0
#define FILEHANDLE_STDOUT   1
#define FILEHANDLE_STDERR   2

/* pipe file descriptors*/
#define FILEHANDLE_PIPE_A_READ 3
#define FILEHANDLE_PIPE_A_WRITE 4
#define FILEHANDLE_PIPE_B_READ 5
#define FILEHANDLE_PIPE_B_WRITE 6
#define FILEHANDLE_PIPE_C_READ 7
#define FILEHANDLE_PIPE_C_WRITE 8
#endif

\end{verbatim}
\subsection{proc/syscall.c}
\begin{verbatim}
  case SYSCALL_RAND:
    V0 = _get_rand((int) A1);
    break;
  case SYSCALL_PIPE:
    V0 = pipe_start((int*) A1);
    break;
  case SYSCALL_PIPE_DESTROY:
    V0 = pipe_destroy((int*) A1);
    break;
  case SYSCALL_DUP:
    V0 = dup((int*) A1);
    break;

\end{verbatim}

\subsection{proc/pipe.h}
\begin{verbatim}
#ifndef BUENOS_PIPE
#define BUENOS_PIPE
#define PIPE_SIZE 100
#define P_SIZE (PIPE_SIZE + 1)

typedef struct pipe_t_struct{
  char *usr_sem;
  int read;
  int write;
  char p[P_SIZE][256];
} pipe_t;

int pipe_init(pipe_t *pipe);
int pipe_write(int file, char* buf, int length);
int pipe_read(int file, char* buf, int length);
int pipe_destroy(int *pipe);

int pipe_start(int *fds);

#endif
\end{verbatim}

\subsection{proc/pipe.c}
\begin{verbatim}
#include "drivers/gcd.h"
#include "fs/vfs.h"
#include "kernel/assert.h"
#include "proc/syscall.h"
#include "lib/debug.h"
#include "proc/pipe.h"
#include "proc/usr_sem.h"
#include <stdarg.h>
#include <stddef.h>
#include "proc/usr_sem.h"
#include "lib/debug.h"
#include "lib/libc.h"
pipe_t pipe_a;
pipe_t pipe_b;
pipe_t pipe_c;
usr_sem_t *pipe_a_sem;
usr_sem_t *pipe_b_sem;
usr_sem_t *pipe_c_sem;

/**
 * Write to the write end of the pipe.
 */
int pipe_write(int file, char* buf, int length) {
    pipe_t *pipe;
    switch(file) {
        case FILEHANDLE_PIPE_A_WRITE:
            pipe = &pipe_a;
            break;
        case FILEHANDLE_PIPE_B_WRITE:
            pipe = &pipe_b;
            break;
        case FILEHANDLE_PIPE_C_WRITE:
            pipe = &pipe_c;
            break;
        default:
            return -1;
    }
    /* Add the new item to the write. */
    int ret = usr_sem_p((*pipe).usr_sem);
    if(ret < 0){
       return -1;
    }
    while(pipe->write == (( pipe->read - 1 + P_SIZE) % P_SIZE)) /* Pipe is Full*/
    {
        usr_sem_v((*pipe).usr_sem);
        usr_sem_p((*pipe).usr_sem);
    }
    stringcopy(pipe->p[pipe->write], buf, length);
    pipe->write = (pipe->write + 1) % P_SIZE;
    usr_sem_v((*pipe).usr_sem);

    return 0;
}

/**
 * read from the read end.
 *
 * Returns: a pointer to the value of the node at the read end of the pipe; NULL
 * if the pipe is empty, or an error occurred.
 */
int pipe_read(int file, char* buf, int length) {
    pipe_t *pipe;
    DEBUG("debuginit", "pipe.c read 1\n");
//    kprintf("file %d\n", file);
    switch(file) {
        case FILEHANDLE_PIPE_A_READ:
            pipe = &pipe_a;
            break;
        case FILEHANDLE_PIPE_B_READ:
            pipe = &pipe_b;
            break;
         case FILEHANDLE_PIPE_C_READ:
            pipe = &pipe_c;
            break;
         default:
            return -1;
    }

    int ret = usr_sem_p((*pipe).usr_sem);
    if(ret < 0){
        return -1;
    }

    DEBUG("debuginit", "pipe.c read 1\n");

    while(pipe->write == pipe->read)         /* Pipe is Empty*/
    {
        usr_sem_v((*pipe).usr_sem);
        usr_sem_p((*pipe).usr_sem);
    }

    stringcopy(buf, pipe->p[pipe->read], length);
    pipe->read = (pipe->read +1) % P_SIZE;
    usr_sem_v((*pipe).usr_sem);
  return length;
}

/**
 * Destroy the pipe after use.
 *
 * put and get must not be called after destroy. Unless init has been called
 * again.
 */
int pipe_destroy(int *pipe) {
    switch(*pipe){
    case 3:
        return usr_sem_destroy(pipe_a_sem);
    case 5:
        return usr_sem_destroy(pipe_b_sem);
    case 7:
        return usr_sem_destroy(pipe_c_sem);
    default:
        return -10;
    }
}

int pipe_start(int *fds) {
    //find a free pipe
    if((pipe_a_sem = usr_sem_open("pipe_a_sem", 0))){
       pipe_a.read = 0;
       pipe_a.write = 0;
       pipe_a.usr_sem = pipe_a_sem;
       *fds = FILEHANDLE_PIPE_A_READ;
       *(fds + 1) = FILEHANDLE_PIPE_A_WRITE;
       usr_sem_v(pipe_a_sem);
       DEBUG("debuginit", "proc/pipe.c pipe_init sem closed\n");

    }
    else{
        if((pipe_b_sem = usr_sem_open("pipe_b_sem", 0))){
           DEBUG("debuginit", "proc/pipe.c pipe_init sem b opened\n");
           pipe_b.read = 0;
           pipe_b.write = 0;
           pipe_b.usr_sem = pipe_b_sem;
           *fds = FILEHANDLE_PIPE_B_READ;
           *(fds + 1) = FILEHANDLE_PIPE_B_WRITE;
           usr_sem_v(pipe_b_sem);
           DEBUG("debuginit", "proc/pipe.c pipe_init sem b closed\n");
        }
        else {
            if((pipe_c_sem = usr_sem_open("pipe_c_sem", 0))){
               DEBUG("debuginit", "proc/pipe.c pipe_init sem b opened\n");
               pipe_c.read = 0;
               pipe_c.write = 0;
               pipe_c.usr_sem = pipe_c_sem;
               *fds = FILEHANDLE_PIPE_C_READ;
               *(fds + 1) = FILEHANDLE_PIPE_C_WRITE;
               usr_sem_v(pipe_c_sem);
               DEBUG("debuginit", "proc/pipe.c pipe_init sem b closed\n");
            }
            else {
                DEBUG("debuginit", "proc/pipe.c pipe_init return -1\n");

                return -1;
            }
        }
    }
    return 0;
}



\end{verbatim}

\subsection{proc/dup.h}
\begin{verbatim}
#ifndef BUENOS_PROC_DUP
#define BUENOS_PROC_DUP

#include "kernel/config.h"
#include "lib/types.h"

#define DUP_ENTRIES 9

int dup(int *fd);
void dup_map_init();
int get_fd(int fd);
#endif

\end{verbatim}

\subsection{proc/dup.c}
\begin{verbatim}
#include "proc/dup.h"
#include "proc/process.h"
#include "proc/elf.h"
#include "kernel/thread.h"
#include "kernel/assert.h"
#include "kernel/interrupt.h"
#include "kernel/sleepq.h"
#include "kernel/config.h"
#include "fs/vfs.h"
#include "drivers/yams.h"
#include "vm/vm.h"
#include "vm/pagepool.h"
#include "lib/types.h"
#include "lib/debug.h"
/**
 * This module contains controls the mapping of file descriptors
 *
 */
int dup_map[DUP_ENTRIES];

/* We need a spinlock to lock accessses to the dup map. */
spinlock_t dup_map_table_slock;

int dup(int *fd)
{
    dup_map[*(fd +1)] = *fd;
    return 0;
}

void dup_map_init()
{
    for(int i = 0; i < DUP_ENTRIES +1; i++){
        dup_map[i] = i;
    }
}

int get_fd(int fd){
    return dup_map[fd];

}


\end{verbatim}

\subsection{proc/io.h}
\begin{verbatim}
int io_read(openfile_t file, void* buffer, int length)
{
  int res;
  //check the dup mapping to se if another fd should be used
  file = get_fd(file);
  switch(file) {
  case FILEHANDLE_STDIN:
    res = tty_read(buffer, length);
    break;

  case FILEHANDLE_STDOUT:
  case FILEHANDLE_STDERR:
    res = VFS_INVALID_PARAMS;
    break;

  case FILEHANDLE_PIPE_A_READ:
    res = pipe_read(file, buffer, length);
    break;

  case FILEHANDLE_PIPE_B_READ:
    res = pipe_read(file, buffer, length);
    break;

  case FILEHANDLE_PIPE_C_READ:
    res = pipe_read(file, buffer, length);
    break;

  default:
    file -= 3;
    if (!process_has_open_file(file)) {
      res = VFS_NOT_OPEN_IN_PROCESS;
    } else {
      res = vfs_read(file, buffer, length);
    }
  }

  return res;
}

int io_write(int file, void* buffer, int length)
{
  int res;
  //check the dup mapping to se if another fd should be used
  file = get_fd(file);
  switch(file) {
  case FILEHANDLE_STDIN:
    res = VFS_INVALID_PARAMS;
    break;

  case FILEHANDLE_STDOUT:
    res = tty_write_stdout(buffer, length);
    break;

  case FILEHANDLE_STDERR:
    res = tty_write_stderr(buffer, length);
    break;

  case FILEHANDLE_PIPE_A_WRITE:
    res = pipe_write(file, buffer, length);
    break;
  case FILEHANDLE_PIPE_B_WRITE:
    res = pipe_write(file, buffer, length);
    break;
  case FILEHANDLE_PIPE_C_WRITE:
    res = pipe_write(file, buffer, length);
    break;

  default:
    file -= 3;
    if (!process_has_open_file(file)) {
      res = VFS_NOT_OPEN_IN_PROCESS;
    } else {
      res = vfs_write(file, buffer, length);
    }
  }

  return res;
}



\end{verbatim}

\subsection{proc/module.mk}
\begin{verbatim}
# Makefile for the kernel module

# Set the module name
MODULE := proc


FILES := exception.c elf.c process.c syscall.c usr_sem.c io.c pipe.c dup.c

SRC += $(patsubst %, $(MODULE)/%, $(FILES))







\end{verbatim}

\subsection{init.main}
\begin{verbatim}
  /*initialise the fildescriptor ma*/
  kwrite("Initializing filedescriptor map\n");
  dup_map_init();

\end{verbatim}


\end{document}
