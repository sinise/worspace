\documentclass[12pt]{article}
\usepackage{amsmath} % flere matematikkommandoer
\usepackage[utf8]{inputenc} % æøå
\usepackage[T1]{fontenc} % mere æøå
\usepackage[danish]{babel} % orddeling
\usepackage{enumerate}

\title{Projekt A}
\author{Sebastian O. Jensen 140679}


\begin{document}


\maketitle
\section{Gruppeopgave 2}
\begin{enumerate}
\item Den transistor baserede computer

Den transistor baserede computer findes i dag rigtig mange steder.  I
smartphones, biler, laptops mv. Den er bygget af mange transistore der fungere som logiske
kontakter der enten kan være 0 eller 1. Den arbejder sekvencielt, hvilket
vil sige at den håndtere en opgave af gangen. Hastigheden hvor med den kan løse en
opgave er afhængig klockfrekvensen. Klokfrekvensen beskriver hvor mange
instruktioner der kan startes i sekunded. Derfor vil computeren blive dobelt så
hurtig når clockfrekvensen fordobles. Antalet af transistore der
kan være i en chip på samme størelse er fordoblet ca. hver
18 måned, hvilket har resulteret i at frem til ca år 2000 er clockfrekvensen
også fordoblet hver 18 måned. Herefter har man i stedet forøget antallet af
kerner i cpu'en så den kan håndtere flere opgaver samtidig. (Supercomputere -
mange bække små, Brian Vinter, DIKU), (Cramming more components. Gordon E.
Moore)

\item Den dna baserede computer

Den dna baserede computer fungerer ved at kode forskellige dna
strenge svarende til nogle atributer i det problem der ønskes løst. Disse dna
strenge forbinder sig so med hinanden. Derved får man lavet alle mulige dna
strenge som kan danes af de givne dna strenge. Det sker simultant. Hvor imod den
transistor baserede computer ville skulle beregne problemerne en efter en. Når
alle de mulige dna strenge er dannet benytter man forskellige metoder til at
frasortere de strenge som ikke matcher den løsning man søger. F.eks. hvis man
skal finde den korteste vej mellem to punkter sorterer man alle strenge fra der
ikke starter og ender med de strenge der representere de to punkter. Derefter
kan man så finde den korteste streng som så repræsentere den korteste vej. Den
dna baserede computer arbejder altså parallelt.



Det er klart at kvante komputeren har et stort potentiale i og med at den
mulighvis vil kunne løse problemer som den traditionele transistor baserede
computer ikke vil være i stand til at løse, ihvertfald ikke på overskuelig tid.
Der hvor kvante computeren virkelig kan gøre en forskel i forhold til i dag er
bl.a. udregning af komplekse analyzer i finansverden, forskning i bilogi,
kryptering, databasesystemer og andre  steder der har at gøre med problemer med
exponentinel vækst. F.eks. vil den kunne hente svar fra en database mange gange
hurtigere end man kan med traditionele databasesystemer. En kvante computer
henter den data man efterspørger ud med det samme da den kan se på alt data på
en gang, hvor i mod den transistorbaserede computer først skal gennemsøge
databasen. Dette vil for store tjenester som google gøre at de kan forbedre
søgning helt utrolig meget. En kvante computer vil kunne løse opgaver mellem
titusinde til flere milioner gange hurtigere end en traditionel
computer(Hardisken 31-10-2014. verdens hurtigeste computer)
Men om den bliver vert mands eje er ikke så sikkert da det er en kompliceret
computer. Samtidig er det svært at se behovet for at vi i vores personlige
devices vil have brug for denne enorme regnekraft. Man kan forstile sig at den
vil blive brugt i skyen til at lave de tunge beregninger. F.eks. Hvis ens
navigations program skal finde den korteste vej mellem København og Moskva 
vil programet sende opgaven til en server som så i løbet af meget kort tid
svare tilbage med den korteste vej. Der ud over er softeware til en
kvantecomputer blablablablabalballbalbalblalbalblalbalblaqlballablalb. Den
transistor baseret computter er en universiel computer som man nemt kan
forbinde til andre cencorer, kammerarer, mv. samtidig er softwaren ikke så
kompliceret som til en kvante computer. Derfor ser vi
ikke at kvantecomputeren udkonkurerer den transistor baserede computer men at de
vil arbejde sammen i støre systemer og løse de opgaver den er specielt god til.
Den dna baserede computer vurdere vi udviklingsmæsigt er langt fra det stadie
som kvantecomputeren er i. Dette konkludere vi ud fra at vi kun har kunnet finde
det ene forsøg som er beskrevet i Computing with DNA by Leonard M. Adleman, men
at der arbejdes på at fremstile kvantecomputere flere steder, bl.a. på Århus
universistet (Hardisken 31-10-2014. verdens hurtigeste com). Teoretisk set
har dna computeren også et stort potentiale da den kan udføre utrolig mange
opgaver simultant, men det er stadig meget tidligt at sige om den vil få nogen
relevant udbredelse
 
\end{enumerate}
\end{document}
