%%This is a very basic article template.
%%There is just one section and two subsections.
\documentclass[12pt]{article}
\usepackage{amsmath} % flere matematikkommandoer
\usepackage[utf8]{inputenc} % æøå
\usepackage[T1]{fontenc} % mere æøå
\usepackage[danish]{babel} % orddeling
\usepackage{enumerate}
\usepackage{tabto}
\title{Projekt C}
\author{Sebastian O. Jensen, GJX 653}

\begin{document}
\newcommand{\A}{
\left(\begin{array}{cc}
6&1\\
4&4\\
2&2\\
13&-2
\end{array}\right)
  }

\newcommand{\AT}{
\left(\begin{array}{cccc}
6&4&2&13\\
1&4&2&-2
\end{array}\right)
  }
 
\maketitle
\newpage
\section{Opgave}
\begin{enumerate}[(a)]
\item 
For at to vektore er ortogonal skal deres prik produkt være nul

$$
\left(\begin{array}{c}
6\\4\\2\\13
\end{array}\right)\cdot
\left(\begin{array}{c}
1\\4\\2\\-2
\end{array}\right)= 6 \cdot 1 + 4 \cdot 4 + 2 \cdot 2 + 13 \cdot \cdot-2= 0
$$
For at vektorene skal være en basis for U kræver det at de er lignært
uafhængige. De er lignært uafhængige hvis rank er lig antalet af vektore.
Vi bringer de 2 vektore på echelon form
$$
\left(\begin{array}{cc}
1&1/6\\
0&1\\
0&0\\
0&0\\
\end{array}\right)
$$
Det ses her af at vektore er lignært uafhængige samtidig har vi oplyst at
vektorene udspænder U. Derfor er de en basis for for U. Samtidig er
prikproduktet 0 or dermed er de også ortogonale.

\item 
Fra bogen har vi følgende
$$
proj_\mu(v)= \left(\begin{array}{cccc}
u_1&
u_2&
\ldots&
u_k
\end{array}\right)
\left(\begin{array}{c}
u^T_1\\
u^T_2\\
\ldots\\
u^T_k
\end{array}\right)v  
$$

Der gælder derfor følgende

$$
Pv= \left(\begin{array}{cccc}
u_1&
u_2&
\ldots&
u_k
\end{array}\right)
\left(\begin{array}{c}
u^T_1\\
u^T_2\\
\ldots\\
u^T_k
\end{array}\right)v  
$$


$$
P= \left(\begin{array}{cccc}
u_1&
u_2&
\ldots&
u_k
\end{array}\right)
\left(\begin{array}{c}
u^T_1\\
u^T_2\\
\ldots\\
u^T_k
\end{array}\right)  
$$

Som kan udtrykes som
$$
P=A(A^T \cdot A)^{-1} A^T
$$
Hvor A er en matrice hvor søjlerne er basis for underrumet U

Dette giver følgende

$$
P= \A \left(\AT \A\right) ^1 \AT =
$$
$$
  \mathbf{P} \,=\, 
    \frac{1}{225}
    \left(\!\!
    \begin{array}{rrrr}
       45 &  60 &  30 &  60 \\
       60 & 160 &  80 & -20 \\
       30 &  80 &  40 & -10 \\
       60 & -20 & -10 & 205 
    \end{array}
    \!\!\right).
$$

Hermed er det vist at projektionsmatricen er givet ved P

\item
Ved at benytte projektionsmatricen fra opgave b bestemes den ortogonale
projektion af vektoren v på underrummet U 

$$
  \mathbf{Pv} \,=\, 
    \frac{1}{225}
    \left(\!\!
    \begin{array}{rrrr}
       45 &  60 &  30 &  60 \\
       60 & 160 &  80 & -20 \\
       30 &  80 &  40 & -10 \\
       60 & -20 & -10 & 205 
    \end{array}
    \!\!\right).
\cdot
\left(\begin{array}{c}
15\\0\\0\\0
\end{array}\right)=
\left(\begin{array}{c}
3\\4\\2\\4
\end{array}\right)
$$
\item
Vi benyter os af Matrice A fra opgave b hvor søjlene er basis for underrummet U
og finder den reducerede rækkeechelon form for $A^T$

$$
\AT=
$$
$$
\left(\begin{array}{cccc}
1&4&2&-2\\
0&1&1/2&-25/20
\end{array}\right)=
$$
$$
\left(\begin{array}{cccc}
1&0&0&3\\
0&1&1/2&-25/20
\end{array}\right)=
$$
Her af aflæses en basis for $U^\perp$

$$
\left(\begin{array}{cc}
0&-3\\
-1/2&25/20\\
1&0\\
0&1
\end{array}\right)
$$
\end{enumerate}
\section{Opgave}
\begin{enumerate}[(a)]
\item 
i følge Gram-Schmit proceduren er $q_1$ og $q_2$ givet ved følgene udtryk

 $$q_1 = \frac{v_1}{||v_1||}$$
 $$q_2 = \frac{v_2- \left(q_1\cdot u_2\right)q1}{||v_2- \left(q_1\cdot
 u_2\right)q1||}$$

Hvor $q_1$ og $q_2$ er en otonormal basis for V

Hvilket giver følgende
$$q_1 =\frac{1}{3}
\left(\begin{array}{c}
-1\\
2\\
2
\end{array}\right)
$$

$$q_1 \cdot v_2 =\frac{1}{3}
\left(\begin{array}{c}
-1\\
2\\
2
\end{array}\right) \cdot
\left(\begin{array}{c}
1\\
1\\
4
\end{array}\right) = 3$$

$$
v_2- \left(q_1 \cdot v_2 \right)q_1 = 
\left(\begin{array}{c}
1\\
1\\
4
\end{array}\right)- 
\left(\begin{array}{c}
-1\\
2\\
2
\end{array}\right) =
\left(\begin{array}{c}
2\\
-1\\
2
\end{array}\right)  
$$

$$
||v_2- \left(q_1\cdot
 v_2\right)q_1|| = 3
 $$
 
$$
q_2 = \frac{1}{3} \left(\begin{array}{c}
2\\
-1\\
2
\end{array}\right) 
$$

\item
I følge Hardys bog kan koordinatvektoren $[w]_\beta$ bestemes med følgende
udtryk

$$
[w]_\beta =
\left(\begin{array}{c}
x_1\\
x_2\\
x_3
\end{array}\right)
=
\left(\begin{array}{c}
\frac{w\cdot u_1}{||u_1||^2}\\
\frac{w\cdot u_2}{||u_2||^2}\\
\frac{w\cdot u_3}{||u_3||^2}
\end{array}\right)
$$

Hvilket giver følgende koordinatvektor

$$
\left(\begin{array}{c}
\frac{3}{1}\\
\frac{2}{1}\\
\frac{1}{1}\\
\end{array}\right)
=
\left(\begin{array}{c}
3\\
2\\
1
\end{array}\right)
$$

\item
$Q^{-1}$ bestemes vha. identitets matricen

$$
\left(\begin{array}{ccc|ccc}
-1&2&2&1&0&0\\
2&-1&2&0&1&0\\
2&2&-1&0&0&1
\end{array}\right)
$$


Der udføres række operationer og vi for følgende

$$
\left(\begin{array}{ccc|ccc}
1&0&0&-1/9&2/9&2/9\\
0&1&0&2/9&-1/9&2/9\\
0&0&1&2/9&2/9&-1/9
\end{array}\right)
$$

$Q^{-1}$ er her med givet ved følgende matrice.

$$
Q^{-1} = \left(\begin{array}{ccc}
-1/9&2/9&2/9\\
2/9&-1/9&2/9\\
2/9&2/9&-1/9
\end{array}\right)
$$
\end{enumerate}
\section{Opgave}
\begin{enumerate}[(a)]
\item
Vi benytter mindste kvadraters methode til at finde forskriften for ln y som y =
mx + c
Først opstiles matrix systemet ud fra de opgivne data og vi får følgende

$
A = \left(\begin{array}{cc}
1&2004\\
1&2005\\
1&2007\\
1&2008\\
1&2009\\
1&2010\\
1&2011\\
1&2012\\
1&2013
\end{array}\right)$
$x = \left(\begin{array}{c}
c\\m
\end{array}\right)
$
$b = \left(\begin{array}{c}
31,890\\
32,550\\
33,801\\
34,564\\
35,104\\
35,481\\
36,891\\
37,331\\
38,061
\end{array}\right)
$

Da rank A = 2 har vi følgende

$$
\bar{x}= \left(\begin{array}{c}
\bar{c}\\
\bar{m}
\end{array}\right)=
\left(A^TA\right)^{-1}A^Tb
$$

Der benyttes en matrix beregner hvilket giver følgende
$$
\bar{x}= \left(\begin{array}{c}
-1343,01\\
0,6860
\end{array}\right)
$$

forskriften for ln y er derfor givet ved følgende rette linje

$$
ln y=0,6860t-1343,01
$$
\item
Fra opgave b har vi

$$
ln y=0,6860t-1343,01
$$

Dette kan omskrives til 


$$
y= e^{0,6860t-1343,01}
$$
$$
y= e^{0,6860 \cdot 2004-1343,01} \cdot e^{0,6860(t-2004)}
$$
$$
y= 6,112854146 \cdot 10^{13} \cdot e^{0,6860(t-2004)}
$$
Hvilket stemmer overens med den i opgaven nævnte funktion. Funktionerne stemmer
ikke helt overens da de udregnede værdier i opgave a ikke er præcis de samme
som opgavens.

\item  
Ved at benytte tilnærmelsen fra opgaven fåes følgende resultat for antal flops i
1996

$$
6.44\cdot10^{13} \cdot e^{0.639(1996-2004)}= 387,94 \cdot 10^{9}
$$

Hvilket er en del mere end det faktiske antal flops i 1996. Hvis parameterne som
blev fundet i opgave a blev benyttet vil udregningen give $266,3 \cdot 10^{9}$
hvilket er nærmere det faktiske niveu i 1996.
\end{enumerate}
\section{Opgave}
\begin{enumerate}[(a)]
ProjectC.java er main programmet der benyter sig den udleveret klasse
Matrix.java. I Matrix. java er udover den udfyldte methode gramschmith
også tilføjet en methode sub som er en modification af add. 
Følgende kode er implementeret i bunden af den udleveret Matrix klasse.
Se java fil for mere overskugelig indentering

-  

  /**
   * Create a new matrix where the column vectors form an orthonormal subspace spanning the same as this, and calculated by the Gram-Schmidt process.
   *
   * @return a new matrixs
   **/
  public Matrix GramSchmidt() {
    int m = rows();
    int n = cols();

    Matrix r = new Matrix(m,n);
	Matrix q = new Matrix(m,n);
    for(int j = 1; j <= n; j++){
        q = q.replaceCol(j, subMatrix(1, j, m, j));
        for(int i = 1; i <= j-1; i++){
            Matrix qi = q.subMatrix(1, i, m, i);
            Matrix uj = subMatrix(1, j, m, j);
            Matrix qj = q.subMatrix(1, j, m, j);
            r.set(i, j, qi.transpose().mul(uj).get(1, 1));
            q = q.replaceCol(j, qj.sub(qi.mul(r.get(i,j))));
        }
        double qjLength = 0;
        for(int i = 1; i <= m; i++){
            qjLength += Math.pow(q.get(i, j), 2);
        }
        qjLength = Math.sqrt(qjLength);
        r.set(j, j, qjLength);
        q = q.replaceCol(j, q.subMatrix(1, j, m, j).mul(1/(r.get(j,j))));
    }
	return q;
  }

  /**
   * Create a new matrix, which is the element-wise subtraction of this with another matrix B.  The two matrices must be of the same size.
   *
   * @param B a matrix
   * @return a new matrix
   * @throws IllegalArgumentException if number of rows and columns differs in this and B
   **/
  public Matrix sub(Matrix B) {
	if((rows()!=B.rows()) || (cols()!=B.cols())) {
      throw new IllegalArgumentException("Number of rows and columns differ");
	}
	Matrix M = new Matrix(rows(), cols());
	double v;

	for(int i = 1; i <= rows(); i++) {
      for(int j = 1; j <= cols(); j++) {
		v = get(i, j);
		M.set(i, j, v-B.get(i, j));
      }
	}

	return M;
  }

Udskriften fra kørsel af ProjecC er følgende

OPGAVE 4 b 

Matrix A = 

[4.0 0.0 -1.0]

[-8.0 -2.0 4.0]

[-1.0 1.0 3.0]

[12.0 0.0 2.0]

Matrix Q

[0.26666666666666666 -0.13333333333333333 -0.2]

[-0.5333333333333333 -0.7333333333333334 0.4]

[-0.06666666666666667 0.5333333333333333 0.8]

[0.8 -0.4 0.39999999999999997]

\end{enumerate}
\end{document}
