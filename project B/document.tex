\documentclass[12pt]{article}
\usepackage{amsmath} % flere matematikkommandoer
\usepackage[utf8]{inputenc} % æøå
\usepackage[T1]{fontenc} % mere æøå
\usepackage[danish]{babel} % orddeling
\usepackage{enumerate}

\title{Projekt B}
\author{Sebastian O. Jensen, GJX 653}

\begin{document}
\maketitle
\newpage
\section{Opgave 1}
\begin{enumerate}[(a)]
\item

Fra den reducerede rækkeechelonform for A har vi at $x_1$ og $x_2$ er
ledende variabler. Vi kan derfor vælge  $x_3 = s$ og $x_4 = t$ hvilket giver
følgende

$x_1 = -s-t$

$x_2 = -s+t$

$x_3 = s$

$x_4 = t$

Dette kan skrives som følgende.

$s(-1,-1,1,0)$ og $t(-1,1,0,1)$

En bassis for nulrummet null A er hermed givet ved følgende vektorer
$$
\left(\begin{array}{c}
-1\\-1\\1\\0
\end{array}\right),
\left(\begin{array}{c}
-1\\-1\\1\\0
\end{array}\right)
$$

\item
En basis for søjlerummet for col A er givet ved søjlerne i den oprindelige
matrix svarende til søjlerne med pivot 1 taller i den reducerede matrix.
Søjlerne med pivot 1 taller i A* er søjle 1 og 2. En basis for søjlerummet col A
er derfor følgende
$$
\left(\begin{array}{c}
-2\\0\\-1\\3
\end{array}\right),
\left(\begin{array}{c}
1\\1\\-1\\3
\end{array}\right)
$$

\item
Da den reducerede matrice har 2 pivot rækker er rank A = r = 3 og dermed er
dimensionen af underummet ker T

$T = 3$

Dimmensionen af underummet ran T for en matrix med n søjler er givet ved

$$ran T = n-r$$
$$ran T = 4-2 = 2$$

\item 
At finde en vektor x som opfylder at $T(x) = v$ gøres ved at løse ligningen
$$T(x) = Ax = v$$

Da jeg ved at en basis for col A er givet ved søjle 1 og 2 må en mulig
løsning være på formen $x = (x_1, x_2 , 0, 0)$
Jeg opstiler følgende matrice og reducere vha. gauss jordan

$$
\left(\begin{array}{cc|c}
-2&1&-1\\
0&1&1\\
-1&-1&-2\\
3&3&6
\end{array}\right)
$$
$$
\left(\begin{array}{cc|c}
1&0&1\\
0&1&1\\
0&0&0\\
0&0&0
\end{array}\right)
$$

En mulig vektor x er derfor givet ved 

$$
x = \left(\begin{array}{cc|c}
1\\
1\\
0\\
0
\end{array}\right)
$$

For at finde en anden vektor y som opfyldder $T(y) = v$ betragter jeg ligningen
$$T(x) = Ax = v$$
Da v svare til den tredje søjle i A vil en mulig matrix for y være
$$ 
y = \left(\begin{array}{cc|c}
0\\
0\\
1\\
0
\end{array}\right)
$$

Da
$$
Ay = \left(\begin{array}{cccc}
-2&1&-1&-3\\
0&1&1&-1\\
-1&-1&-2&0\\
3&3&6&0
\end{array}\right) \cdot
\left(\begin{array}{c}
0\\
0\\
1\\
0
\end{array}\right)=
\left(\begin{array}{c}
-2 \cdot 0 + 1 \cdot 0 - 1 \cdot 1 -3 \cdot 0\\
0 \cdot 0 + 1 \cdot 0 + 1 \cdot 1 -1 \cdot 0\\
-1 \cdot 0 - 1 \cdot 0 -2 \cdot 1 + 0 \cdot 0\\
3 \cdot 0 + 3 \cdot 0 + 6 \cdot 1 + 0 \cdot 0\\ 
\end{array}\right)=
\left(\begin{array}{cc|c}
-1\\
1\\
-2\\
6
\end{array}\right) = v
$$


\end{enumerate}
\section{Opgave 2}
\begin{enumerate}[(a)]
\item 
For at finde basisskift-matricen  $\mathbf{P}_{\mathcal{B} \leftarrow\,\mathcal{C}}$
opstiles total matrrix T ud fra de givne vektore
$$
T = \left(\begin{array}{cc|cc}
1&1&3&8\\
2&0&4&10\\
3&2&8&21
\end{array}\right)
$$
Ved at benyter gauss jordan fremkomer følgende reducerede matrix.
$$
T = \left(\begin{array}{cc|cc}
1&0&2&5\\
0&1&1&3\\
0&0&0&0
\end{array}\right)
$$

Basisskift-matricen $\mathbf{P}_{\mathcal{B} \leftarrow\,\mathcal{C}}$ er derfor
følgende

$$\mathbf{P}_{\mathcal{B} \leftarrow
\,\mathcal{C}} = 
\left(\begin{array}{cc}
2&5\\
1&3\\
\end{array}\right)
$$
\item
Jeg benytter samme methode til at finde basisskift-matricen
$\mathbf{P}_{\mathcal{C} \leftarrow\,\mathcal{B}}$
$$
T = \left(\begin{array}{cc|cc}
3&8&1&1\\
4&10&2&0\\
8&21&3&2
\end{array}\right)
$$

$$
T = \left(\begin{array}{cc|cc}
1&0&3&-5\\
0&1&-1&2\\
0&0&0&0
\end{array}\right)
$$
Hvilket giver følgende basisskift matrix
$$\mathbf{P}_{\mathcal{C} \leftarrow
\,\mathcal{B}} = 
\left(\begin{array}{cc}
3&-5\\
-1&2\\
\end{array}\right)
$$
\item

Da x kan skrives som lignaer kombinationen $x = 3u_1 + u_2$ gælder det at

$$[x]c = \left(\begin{array}{c}
3\\
1\\
\end{array}\right)
$$

Vi benytter den tidligere beregnet basiskift matrix
$\mathbf{P}_{\mathcal{C}\leftarrow \,\mathcal{B}}$
og får følgende
$$
[x]b =\mathbf{P}_{\mathcal{C}\leftarrow \,\mathcal{B}} \cdot [x]c = 
\left(\begin{array}{cc}
3&-5\\
-1&2\\
\end{array}\right) \cdot
\left(\begin{array}{c}
3\\
1\\
\end{array}\right) =
\left(\begin{array}{c}
4\\
-1\\
\end{array}\right)
$$

Vektoren w udtrykt som en lignaer kombination af $v_1$ og $v_2$ er derfor
følgende $$w = 4v_1 - v_2$$ 
\end{enumerate}

\section{Opgave 3}
\begin{enumerate}[(a)]
\item
Når rumskibet rynker en plads frem vil rumskibets centrum C rykke til rumskibets
spids S da S netop ligger et ryk frem for C.
Samtidig får vi oplyst at punktet $C^F$ er rumskibets centrum efter et ryk frem,
hvilket jo netop var hvad punktet S er derfor gælder det 
$$C^F = S$$
punktet $S^F$ er rumskibets spids efter et ryk frem. Dette svare til rumskibets
spids forskudt et ryk i rumskibets længederetning. Dette kan udtrykes
som følgende
$$ S^F = S + (S - C)$$
$$ S^F = 2S - C)$$
Hvilket også kan skrives på vektor form
$$ \left(\begin{array}{c}
s_1^F\\
s_2^F
\end{array}\right) =
 \left(\begin{array}{c}
s_1\\
s_2
\end{array}\right) +
 \left(\begin{array}{c}
s_1\\
s_2
\end{array}\right) - 
 \left(\begin{array}{c}
c_1\\
c_2
\end{array}\right) = 
 \left(\begin{array}{c}
2s_1-c_1\\
2s_2-c_2
\end{array}\right)
$$
\item
Fra opgave a har vi følgende.
$$c^F_1 = s_1$$
$$c^F_2 = s_2$$
$$s^F_1 = 2s_1-c_1$$
$$s^F_2 = 2s_2-c_2$$

Her ud fra kan nu opstiles følgende

$$
\left(\begin{array}{c}
c_1^F\\
c_2^F\\
s_1^F\\
s_2^F
\end{array}\right)
 = \left(\begin{array}{cccc}
0&0&1&0\\
0&0&0&1\\
-1&0&2&0\\
0&-1&0&2
\end{array}\right)
\left(\begin{array}{c}
c_1\\
c_2\\
s_1\\
s_2
\end{array}\right)
$$
F er derfor gevet ved følgende matrix
$$ F  = \left(\begin{array}{cccc}
0&0&1&0\\
0&0&0&1\\
-1&0&2&0\\
0&-1&0&2
\end{array}\right)
$$

\item
$$cos(20) = 0,94$$
$$sin(20) = 0,34$$

\item Hvilket giver følgende matricer
$$\mathbf{L}_\theta =
  \left(\begin{array}{cccc}
    1 & 0 & 0 & 0 \\
    0 & 1 & 0 & 0 \\
    0,6 & 0,34 & 0,94 & -0,34 \\
    -0,34 & 0,6 & 0,34 & 0,94
\end{array}\right)
$$$$
  \mathbf{R}_\theta =
  \left(\begin{array}{cccc}
    1 & 0 & 0 & 0 \\
    0 & 1 & 0 & 0 \\
    0,6 & -0,34 & 0,94 & 0,34 \\
    0,34 & 0,6 & -0,34 & 0,94
  \end{array}\right)
$$
hvor $\theta$ er vinklen på 20 grader

Jeg benytter mul fra den udleveret java klasse og ganger matricerne sammen fra
højre mod venstre.
$$RFLFFR\left(\begin{array}{c}
0\\
0\\
0\\
1
\end{array}\right) \approx =
\left(\begin{array}{c}
1,05\\
3,89\\
2,64\\
7,77
\end{array}\right)
$$

Efter tastekombinationen er udført vil rumskiber have følgende position
Rumskibets centrum vil være i 
$$(1,05. 3,89)$$
Og rumskibets spids vil være i
$$(2,64. 7,77)$$
\end{enumerate}

\section{Opgave 4}
\begin{enumerate}[(a)]
\item
Følgende kode er implementeret i klassen Matrix. Se kildekoden for mere
overskugelig indentering.

  /**

   * Create a new matrix which is the matrix product of this with B.

   *

   * @param B a matrix with same number of rows as this has columns

   * @return a new matrix of size rows() x B.cols()

   * @throws IllegalArgumentException if cols() is different from B.r$

   **/

  public Matrix mul(Matrix B) {

    Matrix M = new Matrix(rows(), B.cols());

    if(cols() != B.rows()){

      System.out.print("number of colums in A has to equal number of $

    }

    else{

      for(int i = 1; i <= M.rows(); i++){

        for(int j = 1; j <= M.cols(); j++){

          double v = 0;

          for(int k = 1; k <= cols(); k++){

            v = v + get(i, k)*B.get(k, j);

          }

          M.set(i, j, v);

        }

      }

    }

    return M;

  }

\item

Til at definerer Matricerne A og B samt udregne AB og BA er lavet en java fil
med navnet ProjectB.java som giver følgende udskrift ved kørsel

Java ProjectB 

Matrix A = 

[1.0 2.0 3.0]

[4.0 5.0 6.0]

 
Matrix B = 

[-7.0 8.0]

[9.0 -10.0]

[-11.0 12.0]

A * B =

[-22.0 24.0]

[-49.0 54.0]


B * A =

[25.0 26.0 27.0]

[-31.0 -32.0 -33.0]

[37.0 38.0 39.0]

\end{enumerate} 
\end{document}